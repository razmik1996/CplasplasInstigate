\hypertarget{index_Introduction}{}\section{Introduction}\label{index_Introduction}
Class dynamic array. You can create aray which can change size as required array has push\+\_\+back function which can add new element from front\hypertarget{index_Testing}{}\section{Testing}\label{index_Testing}
Step 1 -\/ Creating dynamic\+Array object$\ast$~\newline
Step 2 -\/ pushing 5 and printing size, capacity and first element~\newline
Step 3 -\/ pushing 8 and printing size and capacity~\newline
Step 4 -\/ pushing 20 and printing size and capacity~\newline
Step 5 -\/ pushing 15 and printing size and capacity~\newline
Step 6 -\/ pushing 13 and printing size and capacity~\newline
Step 7 -\/ pushing tree 3 and printing size, capacity and elements of dynamic\+Array~\newline
Step 8 -\/ poping element and print size, capacity and elements of dynamic\+Array~\newline
Step 9 -\/ shrinking to fit printing size, capacity and elements of dynamic\+Array~\newline
 \hypertarget{index_Installation}{}\section{Installation}\label{index_Installation}
step 1\+: type make~\newline
step 2\+: run program ./bin/class\+Experiments 